%Our work was motivated by the application of information retrieval and relation extraction in textual entailment.
%%Especially, it has been argued that by discovering relations between entities can give valuate information for doing natural logic inference, which in turn can support solving the textual entailment task.
%While most of the current information extraction systems work by extracting all possible entities and their relations from a large text collection, we have not seen any successful application of the large scale existential knowledge acquisition efforts to textual entailment. In this paper we propose an efficient approach that can explore the background knowledge to identify the relations between entities. We focus here on the ancestor and cousin relation which are proved to be very important in natural logic inference. Our approach makes use of Wikipedia and its category structure serving as a conceptual network to quickly identify the relation between entities and also retrieve their class if possible. We show that our approach significantly out perform systems that use existing large scale data as knowledge source. Compare to a baseline system that employs the extended WordNet database, our system significantly improve over 29\% in the average F-score on the relation identification task.

Our work was motivated by the need to support textual inference. This is a knowledge intensive task, but current knowledge acquisition efforts have not addressed the problem in a way that allows inference systems to use it, in tasks such as textual entailment.
While most of the current information extraction systems work by extracting possible entities that appear in close proximity and their relations, from a large text collection, we have not seen any successful application of these large scale existential knowledge acquisition efforts to textual inference. In this paper, we propose an efficient approach that identifies a "knowledge need" on the fly. We focused on two key relations -- the ancestor and the cousin relations -- which have been shown to have a principle role in supporting compositional inference. Refining this is key future direction.  
Our approach makes use of Wikipedia and its category structure, which serves as a conceptual network and allow a quick identification of the relation between entities and the relevant class membership. We show that our approach significantly outperform systems that use existing large scale data as knowledge source.
%Comparing with a baseline system that employs the extended WordNet database, our system significantly improves the average F-score of the relation identification task, by over 29\%.
Furthermore, we show that even when all entities are covered by the extended WordNet used in the baseline, our system still significantly outperforms the baseline system.   