We studied an important component of many computational linguistics
tasks -- identifying taxonomic relations between pairs of concepts. We
argued and provided experimental evidences that static structured
knowledge bases cannot support this task well enough. We developed
TREI, a novel algorithm that leverages information from existing
knowledge sources and uses machine learning and a constraint-based
inference model to get around the noise and the level of uncertainty
inherent in these resources. The experimental study showed that TREI
is significantly better than other systems which were built upon
existing well-known structured resources. Our approach generalized
well across semantic classes and handled well non-Wikipedia
concepts. Our future research will include an evaluation of TREI in
the context of textual inference applications.

\ignore{The key lesson from the success of our approach has to do with
  our combined learning and global inference approach. Furthermore, we
  demonstrate an effective approach to leveraging existing knowledge
  bases for this inference process.}

\ignore{Our machine learning approach makes use of Wikipedia resource,
  but views it as an open and noisy resource; this is augmented with a
  novel use of a constraint-based inference model that allows us to
  make these decisions more robust.}  \ignore{The key technical step
  needed to improve our method further is to better generate concepts
  that are related to the target concepts, so that our global
  inference method becomes even more effective.}

%%% Local Variables:
%%% mode: latex
%%% TeX-master: "jupiter"
%%% End:
